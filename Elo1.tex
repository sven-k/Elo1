%%%%%%%%%%%%%%%%%%%%%%%%%
% Dokumentinformationen %
%%%%%%%%%%%%%%%%%%%%%%%%%
\newcommand{\titleinfo}{Elektronik 1 - Formelsammlung (gem\"ass Unterricht Guido Keel HS12/13)}
\newcommand{\authorinfo}{A.Waldvogel}
\newcommand{\versioninfo}{powered by \LaTeX}
%
%%%%%%%%%%%%%%%%%%%%%%%%%%%%%%%%%%%%%%%%%%%%%
% Standard projektübergreifender Header für
% - Makros 
% - Farben
% - Mathematische Operatoren
%
% DORT NUR ERGÄNZEN, NICHTS LÖSCHEN
%%%%%%%%%%%%%%%%%%%%%%%%%%%%%%%%%%%%%%%%%%%%%
\include{header/header}

% Möglichst keine Ergänzungen hier, sondern in header.tex
\begin{document}

\newpage
	\section{Operationsverstärker}
		\subsection{Opamp Schaltungen (allgemein gilt: $V_{out} = (V_+ - V_-)\cdot A_{ol}$)}
		\subsubsection{Invertierender Verstärker}
			\begin{minipage}[T]{13cm}
                Closed-Loop Verst\"arkung
                \hspace{3mm}\fbox{$V_{out} = \frac{-A_{ol}\cdot\frac{R_2}{R_1+R_2}}{1+A_{ol}\cdot \frac{R_1}{R_1+R_2}}\cdot V_{in} \cong -\frac{R_2}{R_1} \cdot V_{in}$}\\
		        inkl. Offsetspannung
		        \hspace{10.2mm}\fbox{$V_{out} = \frac{A_{ol}}{1+A{ol}\cdot \frac{R_1}{R_1+R_2}}\cdot \left[(V_{AGND}+V_{OS})-V_{in} \cdot \frac{R_2}{R_1+R_2}\right]$}\\
                f\"ur $A{ol}$ gross
                \hspace{22mm}\fbox{$V_{out} = \frac{R_1+R_2}{R_1}\cdot (V_{AGND}+V_{OS})-\frac{R_2}{R_1}\cdot V_{in}$}\\
           	    Ausgangswiderstand    \hspace{10.5mm}\fbox{$r_{out}=0\Omega$}\\
            	Eingangswiderstand    \hspace{11mm}\fbox{$r_{in}=R_1$}
            \end{minipage}
			\begin{minipage}{6cm}
            	\includegraphics[width=6cm]{./bilder/i-verstaerker.png}
            \end{minipage}\\
\hrule

		\subsubsection{Nichtinvertierender Verstärker}
			\begin{minipage}[T]{13cm}
            	
                Closed-Loop Verst\"arkung
                \hspace{3mm}\fbox{$V_{out} = \frac{A_{ol}}{1+A_{ol}\cdot \frac{R_1}{R_1+R_2}} \cdot V_{in} \cong \left( 1+\frac{R_2}{R_1}\right)  \cdot V_{in}$}\\
                inkl. Offsetspannung
                \hspace{10.2mm}\fbox{$V_{out} = \frac{A_{ol}}{1+A_{ol}\cdot\frac{R_1}{R_1+R_2}}\cdot(V_{in}+V_{OS})$}\\
                f\"ur $A{ol}$ gross
                \hspace{22mm}\fbox{$V_{out} =\left(1 + \frac{R_2}{R_1}\right) \cdot (V_{in}+V_{OS})$}\\
            	Ausgangswiderstand    \hspace{10.5mm}\fbox{$r_{out}=0\Omega$}\\
            	Eingangswiderstand    \hspace{11mm}\fbox{$r_{in}=\infty$}\\
            \end{minipage}
			\begin{minipage}{6cm}
            	\includegraphics[width=6cm]{./bilder/ni-verstaerker.png}
            \end{minipage}\\
\hrule

            \subsubsection{Invertierender Addierer}
            \begin{minipage}[T]{13cm}
             Closed-Loop Verst\"arkung
             \hspace{3mm}\fbox{$V_{out}= -R_F \cdot \left(V_{in1}\cdot \frac{1}{R_1} + V_{in2}\cdot \frac{1}{R_2}+ \ldots\right) $}\\
             \hspace*{43mm}\fbox{$A_{CL1}=- \frac{R_F}{R_1}$}
             ;\hspace{0.2mm} \fbox{$A_{CL2}=- \frac{R_F}{R_2}$}
             ; \ldots
            \end{minipage}
            \begin{minipage}{6cm}
                \includegraphics[width=6cm]{./bilder/invertadd.png}
            \end{minipage}\\
            
		\subsubsection{Verstärker mit mehreren Eingängen}
			\begin{minipage}[T]{13cm}
                Closed-Loop Verst\"arkung
                \hspace{3mm}\fbox{$V_{out}=-\frac{R_F}{R_1}\cdot V_{in1}-\frac{R_F}{R_2}\cdot V_{in2}+\frac{R_F+(R_1//R_2)}{(R_1//R_2)}\cdot V_{in3}$}\\
            	\hspace*{43mm}\fbox{$A_{CL1}=-\frac{R_F}{R_1}$}
            	;\hspace{0.3mm}\fbox{$A_{CL2}=-\frac{R_F}{R_2}$}
            	;\hspace{0.3mm}\fbox{$A_{CL3}=\frac{R_F+(R_1//R_2)}{(R_1//R_2)}$}\\
            \end{minipage}
			\begin{minipage}{6cm}
            	\includegraphics[width=6cm]{./bilder/3-eingaenge.png}
            \end{minipage}\\
\hrule

		\subsubsection{Mehrfach-Addierer-Subtrahierer} 		
        \begin{minipage}[T]{13cm}
            1. Man w\"ahlt $R_{F}$\\
            2. Man w\"ahlt $R_{P}$, wobei oft $R_{P}=R_{F}$ gesetzt wird. (optional)\\
            3. $R_{n}=\frac{R_{F}}{\left|A_{n}\right|}$ oder
            $R_{n}=\frac{R_{P}}{\left|A_{n}\right|}$\\ 
            4. Verst\"rkungsbedingung: $A_{N1} +
            \ldots + A_{Nn} = A_{P1} + \ldots + A_{Pn}$ \\Falls unerf\"ullt, muss ein Dummyeingang hinzugefügt werden!
        \end{minipage}
        \begin{minipage}{6cm}
            \includegraphics[width=5.5cm]{./bilder/mehrfach-addierer-subtrahierer.png} 
        \end{minipage}\\
\newpage
        
		\subsubsection{Gewichteter Subtrahierer}
			\begin{minipage}[T]{13cm}
                Closed-Loop Verst\"arkung
                \hspace{3mm}\fbox{$V_{out}=\frac{R_3}{R_2+R_3}\left(1+\frac{R_F}{R_1}\right)\cdot V_{in2}-\frac{R_F}{R_1}\cdot V_{in1}$}\\
                f\"ur $R_3 = R_F$ und $R_2 = R_1$
                \hspace{0.1mm} \fbox{$V_{out}=\frac{R_F}{R_1}\cdot (V_{in2}-V_{in1})$}
            \end{minipage}
			\begin{minipage}{6cm}
            	\includegraphics[width=6cm]{./bilder/gewichtsub.png}
            \end{minipage}\\		
\hrule

		\subsubsection{Differenzverstärker}
            \begin{minipage}[T]{14cm}
                Closed-Loop Verst\"arkung
                \hspace{3mm}\fbox{$V_{out} = \frac{R_3+R_4}{R_3}\cdot \left( \frac{R_1}{R_1+R_2}\cdot V_{ref} + \frac{R_2}{R_1+R_2}\cdot V_{in1}\right)-\frac{R_4}{R_3}\cdot V_{in2} $}\\
                f\"ur $R_1 = R_3$ und $R_2 = R_4$
                \hspace{1mm} \fbox{$V_{out}= V_{ref} +\frac{R_4}{R_3}\cdot(V_{in1}-V_{in2})$}
            \end{minipage}
            \begin{minipage}{5cm}
                \includegraphics[width=5cm]{./bilder/differenzver.png}
            \end{minipage}\\
\hrule
        
		\subsubsection{Instrumentenverstärker}
		    \begin{minipage}[T]{13cm}
		        Stufe 1 (Pos Eingang)
                \hspace{8.3mm}\fbox{$V_{opo2} = V_{in2}+ \frac{R_{f2}}{R_G}\cdot(V_{in2}-V_{in1})$}\\
                Stufe 1 (Neg Eingang)
                \hspace{8mm}\fbox{$V_{opo1} = V_{in1}- \frac{R_{f1}}{R_G}\cdot(V_{in2}-V_{in1})$}\\
                Closed-Loop Verst\"arkung
                \hspace{3.4mm}\fbox{$V_{out} = V_{ref}+\frac{R_4}{R_3}\cdot\left(1+\frac{R_{f1}+R_{f2}}{R_G} \right)\cdot (V_{in1}-V_{in2}) $}
		    \end{minipage}
		    \begin{minipage}{6cm}
          	    \includegraphics[width=6cm]{./bilder/Instrumentationsverstaerker.png} 
            \end{minipage}\\
\hrule

        \subsubsection{Integrator}
            \begin{minipage}[T]{13cm}
                Closed-Loop Verst\"arkung
                \hspace{3mm}\fbox{$V_{out} =-\frac{1}{R\cdot C} \int{V_{in}}dt + V_{out\hspace{1mm}Anfang} $}\\
                \hspace*{42mm} \fbox{$\frac{dv_{out}}{dt}=-\frac{v_{in}}{RC}$}\\
            \end{minipage} 
            \begin{minipage}{6cm}
                \includegraphics[width=6cm]{./bilder/integrator.png} 
            \end{minipage}\\		
\hrule

		\subsubsection{Differentiator}
        \begin{minipage}[T]{13cm}
            Beim Differentiator gilt $V_{out}=v_N-i_1 \cdot R_F$ wobei $v_N=0$\\
            Closed-Loop Verst\"arkung
            \hspace{3mm}\fbox{$V_{out}=-R_F\cdot C_1 \frac{dv_{in}}{dt}$}\\
            \hspace*{43.2mm}\fbox{$i_1=C_1 \cdot \frac{dv_C}{dt}$}\\
            Die Elemente $C_F$ und $R_1$ sind optional. \\
            Sie beheben jedoch Probleme die ohne \\
            sie entstehen (siehe elemenarer Differentieator). \\
            Mit $C_F$ und $R_1$: \\
            - Keine differentiation bei hoeheren Frequenzen. \\
            - Limitierte Verstaerkuing bei hoeheren Frequenzen. \\
            - Eingangswiderstand immer groesser $R_1$ \\ 
            $\rightarrow$ keine Belastung der Signalquelle\\            
        \end{minipage}
        \begin{minipage}{6cm}
            \includegraphics[width=6cm]{./bilder/differentiator.png}
        \end{minipage}\\
\newpage

        \subsubsection{Buffer (Spannungsfolger / Impedanzwandler)}
            \begin{minipage}[T]{13cm}
                Closed-Loop Verst\"arkung
                \hspace{3mm}\fbox{$V_{out} = \frac{A_{ol}}{1+A_{ol}}\cdot V_{in}\cong V_{in} $}\\
                inkl. Offsetspannung
                \hspace{10.2mm}\fbox{$V_{out} = \frac{A_{ol}}{1+A_{ol}}\cdot(V_{in}+V_{os})\cong V_{in}+V_{os}$}
            \end{minipage} 
            \begin{minipage}{6cm}
                \includegraphics[width=6cm]{./bilder/buffer.png} 
            \end{minipage}\\		
\hrule

	\subsection{Schmitt-Trigger}
        \subsubsection{Nicht invertierender Schmitt-Trigger}
			\begin{minipage}[T]{13cm}
                obere Schaltschwelle
                \hspace{10.8mm}\fbox{$V_{T+} = V_{ref}\cdot \frac{R_1+R_f}{R_f}-V_{out_{min}}\cdot\frac{R_1}{R_f}$}\\
                untere Schaltschwelle
                \hspace{9.6mm}\fbox{$V_{T-} = V_{ref}\cdot \frac{R_1+R_f}{R_f}-V_{out_{max}}\cdot\frac{R_1}{R_f}$}\\
                Hysteresespannung
                \hspace{12.8mm}\fbox{$V_{H} = V_{T+}-V_{T-} = (V_{out_{max}}-V_{out_{min}})\cdot\frac{R_1}{R_f}$}\\
            \end{minipage} 
            \begin{minipage}{6cm}
                \includegraphics[width=6cm]{./bilder/n-schmitt.png} 
            \end{minipage}\\
            
        \subsubsection{Invertierender Schmitt-Trigger}
            \begin{minipage}[T]{13cm}
                obere Schaltschwelle
                \hspace{10.8mm}\fbox{$V_{T+} = \frac{V_{ref}\cdot R_f+ V_{out_{max}}\cdot R_1}{R_1+R_f}$}\\
                untere Schaltschwelle
                \hspace{9.6mm}\fbox{$V_{T-} = \frac{V_{ref}\cdot R_f+ V_{out_{min}}\cdot R_1}{R_1+R_f}$}\\
                Hysteresespannung
                \hspace{12.8mm}\fbox{$V_{H} = V_{T+}-V_{T-} = (V_{out_{max}}-V_{out_{min}})\cdot\frac{R_1}{R_1+R_f}$}\\
            \end{minipage} 
            \begin{minipage}{6cm}
                \includegraphics[width=6cm]{./bilder/i-schmitt.png} 
            \end{minipage}\\
            
        \subsubsection{Invertierender Schmitt-Trigger mit freien Schwellen}
        \begin{minipage}[T]{13cm}
            Schaltschwelle
            \hspace{20.8mm}\fbox{$V_{opp} =V_{ref}\cdot \frac{(R_f//R_0)}{R_1 +(R_f//R_0)}+V_{out}\cdot\frac{(R_1//R_0)}{R_f+(R_1//R_0)}$}\\
            
            Dimensionierung des Schmitt-Triggers: (Gegeben: $V_{ref}$, $V_{T+}$, $V_{T-}$)\\
            1. $V_{out_{max}}$ und $V_{out_{min}}$ ermitteln aus Datenblatt (meistens $V_{DD}$, $GND$)\\
            2. $R_f$ w\"ahlen: typisch $100 k\Omega$\\
            3. Widerst\"ande $R_1$ und $R_0$ dimensionieren\\
        \end{minipage} 
        \begin{minipage}{6cm}
            \includegraphics[width=6cm]{./bilder/i-schmittFreieSchwellen.png} 
        \end{minipage}\\
\hrule

	\subsection{Nichtidealit\"aten des Operationsverst\"arkers}
        \subsubsection{Eingangsstromkompensation (Bias-Strom) ohne AC-Zweige (Kondensatoren)}
        \begin{minipage}[b]{6cm}
            \includegraphics[height=3cm]{./bilder/spannungsfolger.png}\\
            {\bf Spannungsfolger}\\ \\
            $R_2$ sei ein gegebener Quellenwiderstand\\
            $R_F$ muss eingefügt werden\\ \\
            \fbox{$R_F=R_2$}
        \end{minipage}\hfill
        \begin{minipage}[b]{6cm}
            \includegraphics[height=3cm]{./bilder/nichtinver}\\
            {\bf Nichtinvertierender Verstärker}\\ \\
            Hier sei der Quellenwiderstand vernachl\"assigbar.\\ 
            $R_2$ muss eingefügt werden\\ \\
            \fbox{$R_2=R_F//R_1$}
        \end{minipage}\hfill
        \begin{minipage}[b]{6cm}
            \includegraphics[height=3cm]{./bilder/inver}\\
            {\bf Invertierender Verstärker}\\ \\ \\ \\
            $R_2$ muss eingefügt werden\\ \\
            \fbox{$R_2=R_F//R_1$}
        \end{minipage}
        \begin{minipage}{9cm}
            \vspace{3mm}
            \textbf{Ausgangsspannungsfehler} auf Grund des nicht kompensierbaren \textbf{Offsetstroms} (nur bei Kompensation):
        \end{minipage}
        \begin{minipage}{9cm}
            \vspace{7mm}
            \hspace{4mm}\fbox{$V_{out \hspace{1mm}E}=R_f\cdot (I_N-I_P) = R_F \cdot \left|I_{OS}\right|$}\\
        \end{minipage}
\newpage

		\subsubsection{Zusammenfassung aller Fehlereinflüsse}
        \begin{minipage}[T]{12.5cm}
            \underline{{\bf Gleichtaktunterdr\"uckung} Common-Mode-Rejection Ratio $CMRR$}\\
            
            Offsetspannung
            \hspace{18.3mm}\fbox{$V_{OS}=\frac{V_{CM}}{CMRR_{lin}}$} mit \fbox{$V_{CM} = V_{opp} = V_{opn}$}\\
            Lineare Definition
            \hspace{14.2mm}\fbox{$CMRR_{lin}=\frac{dV_{CM}}{dV_{OS}}=10^{\frac{CMRR_{dB}}{20}}$}\\
            Logarithmische Definition
            \hspace{2.2mm}\fbox{$CMRR_{dB}=20 \cdot log(CMRR_{lin})=20 \cdot log\left( \frac{dV_{CM}}{dV_{OS}}\right) $}\\
            Ausgangs-Fehlerspannung
            \hspace{2.2mm}\fbox{$V_{out \hspace{1mm} E}=\left|V_{OS}\right| \cdot A_{CL+}=\left| \frac{\Delta V_{CM}}{CMRR_{lin}}\right|A_{CL+}$}
            
            \vspace{2mm}
            \underline{{\bf Power-Supply-Fehler} Power-Supply-Rejection Ratio $PSRR$}\\
            
            Offsetspannung
            \hspace{18.3mm}\fbox{$V_{OS}=\frac{\Delta V_{Supply}}{PSRR_{lin}}$}\\
            Lineare Definition
            \hspace{14.2mm}\fbox{$PSRR_{lin}=\frac{dV_{Supply}}{dV_{OS}}=10^{\frac{PSRR_{dB}}{20}}$}\\
            Logarithmische Definition
            \hspace{2.2mm}\fbox{$PSRR_{dB}=20\cdot log(PSRR_{lin})=20\cdot log\left( \frac{dV_{Supply}}{dV_{OS}}\right) $}\\
            Ausgangs-Fehlerspannung
            \hspace{2.2mm}\fbox{$V_{out \hspace{1mm}E}=\left|V_{OS}\right| \cdot A_{CL+}=\left| \frac{\Delta V_{Supply}}{PSRR_{lin}}\right| A_{CL+}$}\\
            
            \vspace{2mm}
            \underline{\bf Gesamtfehlerspannung}\\
            
            \hspace*{43.3mm}\fbox{$V_{out\hspace{1mm}E\hspace{1mm}total}=A_{CL+}\cdot\left[\left|V_{OS}\right|+\frac{\left|V_{CM}\right|}{CMRR}+\frac{\left|\Delta V_{Supply}\right|}{PSRR}\right]+\left|I_{OS}\right|\cdot R_F$}\\
        \end{minipage}
        \begin{minipage}[b]{6.5cm}
            \includegraphics[width=6.5cm]{./bilder/OPAmpAlleFehler}
        \end{minipage}
\hrule
\hrule

	\section{Halbleiter}
        \subsection{Diode}
            \underline{\bf Temperaturverhalten}\\
            Thermospannung
            \hspace{15.3mm}\fbox{$V_T = \frac{k_B\cdot T}{q}$} $V_{T_{23^\circ C}} = 25.5 mV$ \fbox{$k_B=1.38065 \cdot 10^{-23}$ J/K; $q=1.6021765 \cdot 10^{-19}$ C; $T$=Temp [K]}\\
            \begin{minipage}[T]{9cm}
                {\bf Sperrbetrieb}\\
                $U_{np} > 5.7V$: Temp-Koeff: $+2 mV/K$\\
                $U_{np} < 5.7V$: Temp-Koeff: $-0.5 mV/K$\\
            \end{minipage}
            \begin{minipage}{9cm}
                {\bf Durchlassbetrieb}\\
                Temp-Koeff: $-2 mV/K$\\
            \end{minipage}
            
            \underline{\bf Strom-Spannungs-Verhalten}\\
            \begin{minipage}[T]{8.5cm}
                Vorw\"artsstrom $I_f$
                \hspace{14.6mm}\fbox{$I_f = I_S\cdot \left(e^\frac{V_f}{m\cdot V_T} -1\right)$}\\
            \end{minipage}
            \begin{minipage}{7cm}
                $I_S$=S\"attigungssperrstrom ($<$ pA)\\
                $m$= Emissionskoeff. ($1<m<2$, meist ca. 1)\\
                $V_T$=Thermospannung
            \end{minipage}
            \begin{minipage}{3.5cm}
                \includegraphics[width=3.5cm]{./bilder/IU_Kennlinie_Diode}\\
            \end{minipage}\\
            
            \begin{minipage}[T]{9.5cm}
                Sperrstrom $I_{SP}$
                \hspace{18mm}\fbox{$I_{SP}(T) = I_{SP}(T_0)\cdot e^{C_R\cdot(T-T_0)}$}\\
                temperaturabh\"angig
                \hspace{11mm}\fbox{$C_{R_{Si}} = \frac{W_g}{2\cdot k_B\cdot T_0^2} \cong 0.07 K^{-1}$}\\
            \end{minipage}
            \begin{minipage}{6cm}
                $C_{R_{Si}}$ (f\"ur Silizium bei Raumtemp.)
            \end{minipage}
            \begin{minipage}{3.5cm}
                \includegraphics[width=3.5cm]{./bilder/IspKennlinieDiodeTemp}\\
            \end{minipage}\\
            
                {\bf Faustregel:} Verdoppelung des Sperrstroms bei einer Temperaturerh\"ohung um $10 ^\circ C$
            
            \begin{minipage}[T]{8.5cm}
                Grosssignalwiderstand
                \hspace{8mm}\fbox{$R_D = \frac{V_0}{I_0}$}\\
                Kleinsignalwiderstand
                \hspace{8.4mm}\fbox{$r_d = \frac{V_T}{I_0}$}\\
            \end{minipage}
            \begin{minipage}{5cm}
                $V_0$= Arbeitspunkt-Spannung\\
                $I_0$= Arbeitspunkt-Strom\\
                $V_T$=Thermospannung
            \end{minipage}
            \begin{minipage}{5.5cm}
                \includegraphics[width=5.5cm]{./bilder/GrossKleinSig}\\
            \end{minipage}
\newpage            

            \begin{minipage}[T]{15.5cm}
                 \underline{\bf Kleinsignalersatzschaltbild}\\
                $R_B$: Bahnwiderstand (Zuleitung, Kontaktierung) : gerader Verlauf der Kennlinie ab $U_f >0.6V$\\
                $r_d$: differentieller Widerstand des pn-\"Ubergangs\\
                $C_D$: Diffusionskapazit\"at bei positiver Diodenspannung\\
                $C_S$: Sperrschichtkapazit\"at bei negativer Diodenspannung\\
            \end{minipage}
            \begin{minipage}{3.5cm}
                \includegraphics[width=3.5cm]{./bilder/DiodeKleinsigErs}\\
            \end{minipage}\\
\hrule
\vspace{1mm}
            \begin{minipage}[t]{9cm}
                \underline{\bf Zener-Diode}\\
                \includegraphics[height=4cm]{./bilder/ZDiodeEig}
            \end{minipage}
            \begin{minipage}[t]{9cm}
                \underline{\bf Kapazit\"atsdiode (Varicap)}\\
                \begin{minipage}[t]{4.9cm}
                    \fbox{$C_S = C_{S0} \cdot\left(1+ \frac{U_{SP}}{U_D}\right)^{-q}$}\\\\
                    $q \cong 0.5$\\
                    $U_{SP}$: gew\"ahlte Sperrspannung\\
                    $C_S$: resultierende Kapazit\"at
                \end{minipage}
                \begin{minipage}{4cm}
                    \includegraphics[width=4cm]{./bilder/CDiodeEig}
                \end{minipage}
            \end{minipage}\\
\hrule
        \subsection{Bipolar-Transistor}
        \subsubsection{Ebers-Moll-Ersatzschaltbild}
            \begin{minipage}[T]{13cm}
                Kollektorstrom
                \hspace{18.7mm}\fbox{$I_C = A_N\cdot I_{SE} \cdot\left(e^{\frac{U_{BE}}{U_T}}-1\right) - I_{SC}\cdot\left( e^{\frac{U_{BC}}{U_T}}-1\right) $}\\
                Emitterrstrom
                \hspace{19.7mm}\fbox{$I_E = I_{SE} \cdot\left(e^{\frac{U_{BE}}{U_T}}-1\right) - A_I\cdot I_{SC}\cdot\left( e^{\frac{U_{BC}}{U_T}}-1\right) $}\\
                Basisstrom
                \hspace{25.1mm}\fbox{$I_B = I_E - I_C$}\vspace{1mm}\\ 
                \hspace*{43mm}$I_{SE}/I_{SC}$: S\"attigungssperrstrom von Emitter/Kollektor\\
                \hspace*{43mm}$U_T$: Thermospannung\\
                \hspace*{43mm}$A_N / A_I$: Verst\"arkungsfaktoren\\
                \hspace*{43mm}$I_{ED}/I_{CD}$: Diodenvorw\"artsstr\"ome (siehe Diode)\\
            \end{minipage}
            \begin{minipage}[T]{6cm}
                \includegraphics[width=6cm]{./bilder/EberMollModell.png}
            \end{minipage}
            
        \subsubsection{Stromverst\"arkungsfaktoren}
            \begin{minipage}[T]{6cm}
                \bf Basisschaltung\\
                \fbox{$A_N = \frac{I_C}{I_E}$}\\
                \includegraphics[height=2.3cm]{./bilder/AmpBasisSch.png}
            \end{minipage}
            \begin{minipage}[T]{6cm}
                \bf Emitterschaltung\\
                \fbox{$B_N = \frac{I_C}{I_B} = \frac{A_N}{1-A_N}$}\\
                \includegraphics[height=2.3cm]{./bilder/AmpEmitSch.png}                
            \end{minipage}
            \begin{minipage}[T]{6cm}
                \bf Kollektorschaltung\\
                \fbox{$C_N = \frac{I_E}{I_B} = \frac{1}{1-A_N}$}\\
                \includegraphics[height=2.3cm]{./bilder/AmpKolSch.png}                
            \end{minipage}
\hrule
        \subsubsection{Kennlinienfelder}
            \begin{minipage}[T]{14cm}
                {\bf Ausgangskennlinie (I)}\\
                typ.: $V_{CE_{sat}}=0.3V$ und Ausgangswiderstand \fbox{$r_{CE} \cong \frac{V_{Early}}{I_{C0}}$}\\
                \hrule\vspace{1mm}
                {\bf Strom\"ubertragungskennlinie (II)}\\
                Kollektorstrom: $I_C = \frac{A_N}{1-A_N}\cdot I_B + \frac{A_N\cdot (1-A_I)}{1-A_N}\cdot I_{SC} = B_N\cdot I_B + I_{CE0} \cong B_N\cdot I_B$\\
                \hrule\vspace{1mm}
                {\bf Eingangskennlinie (III)}\\
                Basisstrom: $I_B = I_{SE}\cdot\left(e^{\frac{V_{BE}}{V_T}}-1\right)$\\
                Kleinsignalwiderstand: \fbox{$r_{BE} = \frac{m\cdot V_T}{I_{B0}}$} mit $I_{B0}$: Arbeitspunktstrom\\
                \hrule\vspace{1mm}
                {\bf Spannungsr\"uckwirkungskennlinie (IV)}\\
                $V_{BE} = \eta \cdot V_{CE}$ mit $\eta \cong 0.1 \%$ (kann meist vernachl\"assigt werden!)\\
            \end{minipage}
            \begin{minipage}[T]{5cm}
                \includegraphics[width=5cm]{./bilder/BipTraKennlinien.png}
            \end{minipage}
\newpage

        \subsubsection{Ersatzschaltung: Transistor als spannungsgesteuerte Stromquelle}
            \begin{minipage}[T]{14cm}
                Steilheit
                \hspace{29.3mm}\fbox{$S = \frac{b}{r_{BE}} = \frac{I_{C0}}{m\cdot V_T} = g_m = \frac{dI_{OUT}}{dV_{IN}}$}\\
                Stromverst\"arkung 
                \hspace{14.7mm}\fbox{$b = h_{FE} = B_N\cdot\left(1+\frac{U_{CE0}}{U_{Early}}\right) \cong B_N$}
            \end{minipage}
            \begin{minipage}[T]{5cm}
                \includegraphics[width=5cm]{./bilder/BipTraErsatzsch.png}
            \end{minipage}
            
            \subsubsection{Kleinsignal-Ersatzschaltung der Emitterschaltung}
            \begin{minipage}[T]{14cm}
                Ausgangsspannung
                \hspace{13mm}\fbox{$U_a = -b\cdot I_B\cdot(R_C // r_{CE})$}\\
                Basisstrom
                \hspace{25.3mm}\fbox{$I_B = \frac{U_e}{r_{BE}}$}\\
                Verst\"arkung
                \hspace{23.3mm}\fbox{$V_U = \frac{U_a}{U_e} = -\frac{b}{r_{BE}}\cdot(R_C//r_{CE}) = -S\cdot(R_C//r_{CE})$}\\
            \end{minipage}
            \begin{minipage}[T]{5cm}
                \includegraphics[width=5cm]{./bilder/KleinSigErsEmmitersch.png}
            \end{minipage}
            
            \subsubsection{Frequenzabh\"angigkeit des Bipolartransistors}
            \begin{minipage}[T]{11cm}
                Stromverst\"arkung
                \hspace{14.5mm}\fbox{$h_{21e}(\omega)\approx \frac{b}{1+ \jmath\omega\cdot r_{BE}\cdot (C_E + C_{BE)}}$}\\
                bei $\omega$ = Transitfrequenz $\omega_T$ ist die Stromverst\"arkung $h_{21e} = 1$\\
                Transitfrequenz wird ebenfalls als Gain-Bandwith-Product GBP\\bezeichnet
            \end{minipage}
            \begin{minipage}[T]{8cm}
                \includegraphics[width=3cm]{./bilder/BipTraFrequenzgang.png}
                \includegraphics[width=5cm]{./bilder/BipTraErsatzschFreq.png}
            \end{minipage}
            
        \subsubsection{Temperaturverhalten von Bipolartransistoren}
            Temperaturabhängigkeit von $V_{BE}: -2mV/K$ und somit Verdoppelung des Sperrstromes bei Temperaturerh\"ohung um $10K$\\
                Stromverst\"arkungsfaktor
                \hspace{4mm}\fbox{$B_N(T) = B_N(T_0)\cdot e^{C_b\cdot(T-T_0)}$ mit $C_b \approx 0.6\% \cdot K^{-1}$}
\vspace{2mm}\hrule

        \subsubsection{Transistor als Schalter}
            \begin{minipage}[T]{13cm}
                Bei der Verwendung des Transistors als Schalter berechnet man den Basisstrom aufgrund des erforderlichen Kollektorstromes und dem vorhandenen Stromverst\"arkungsfaktor und vergr\"ossert den Basisstrom dann anschliessend noch um den Faktor 2 bis 5. Dadurch wird erreicht, dass der Transistor auf jeden Fall v\"ollig durchgeschaltet wird.
            \end{minipage}
            \begin{minipage}[T]{6cm}
                \includegraphics[height=3cm]{./bilder/BipTrAlsSchalterKennl.png}
                \includegraphics[height=3cm]{./bilder/BipTrAlsSchalter.png}
            \end{minipage}
            
        \subsubsection{Gegenkopplungsschaltungen zur Reduktion der Abh\"angigkeit von Temperatur und Tolleranzen}
            \begin{minipage}[T]{5.4cm}
                \includegraphics[height=3cm]{./bilder/Spannungsgegenkopplung.png}
            \end{minipage}
            \begin{minipage}[T]{4cm}
                Gegenkopplung durch $R_2$
            \end{minipage}
            \vrule \hspace{0.1cm}
            \begin{minipage}[T]{5.5cm}
                \includegraphics[height=3cm]{./bilder/Stromgegenkopplung.png}
            \end{minipage}
            \begin{minipage}[T]{4cm}
                Gegenkopplung durch $R_4$\\\\
                $V_{out} = -\frac{R_3}{R_4}\cdot V_{in}$\\\\
                Verst\"arkung viel kleiner als ohne Gegenkopplung
            \end{minipage}
            
        \subsubsection{Transistorverst\"arkerschaltungen}
            \begin{minipage}[T]{4.7cm}
                Emitterschaltung mit\\
                Stromgegenkopplung\\
                \includegraphics[height=3cm]{./bilder/BipTraEmitterschStGk.png}
            \end{minipage}
            \begin{minipage}[T]{4.7cm}
                Emitterschaltung mit\\
                Spannungsgegenkopplung\\
                \includegraphics[height=3cm]{./bilder/BipTraEmitterschSpGk.png}
            \end{minipage}
            \begin{minipage}[T]{5.7cm}
                Basisschaltung\\\\
                \includegraphics[height=3cm]{./bilder/BipTraBasissch.png}
            \end{minipage}
            \begin{minipage}[T]{3.7cm}
                Kollektorschaltung\\\\
                \includegraphics[height=3cm]{./bilder/BipTraKollektorsch.png}
            \end{minipage}
\newpage
         \subsubsection{Differenzverst\"arker}
             \begin{minipage}[T]{15cm}
                 Differenz-Verst\"arkung
                 \hspace{8.4mm}\fbox{$v_d = \frac{U_{A_d}}{U_{E_d}} = -S \cdot (R_C//r_{CE}) \approx -S\cdot R_C$} ; S = Steilheit\\
                 Gleichtaktverst\"arkung
                 \hspace{8.1mm}\fbox{$v_{gl} = -\frac{R_C}{2\cdot R_E}$}\\
                 Gleichtaktunterdr\"uckung
                 \hspace{3.8mm}\fbox{$G = \frac{v_d}{v_{gl}} = 2\cdot S \cdot R_E$}
             \end{minipage}
             \begin{minipage}{4cm}
                 \includegraphics[height=3cm]{./bilder/BipTraDiffAmp.png}
             \end{minipage}\\
             
         \subsubsection{Leistungsendstufen}
             \begin{minipage}[T]{4.7cm}
                 Klasse A-Verst\"arker\\
                 \includegraphics[height=4cm]{./bilder/KlassA_Amp.png}
             \end{minipage}
             \begin{minipage}[T]{4.7cm}
                 Klasse B-Verst\"arker\\
                 \includegraphics[height=4cm]{./bilder/KlassB_Amp.png}
             \end{minipage}
             \begin{minipage}[T]{4.7cm}
                 Klasse AB-Verst\"arker\\
                 \includegraphics[height=4cm]{./bilder/KlassAB_Amp.png}
             \end{minipage}
             \begin{minipage}[T]{4.7cm}
                 Arbeitspunkte jeweiler Klassen\\\\
                 \vspace{5mm}
                 \includegraphics[width=4.7cm]{./bilder/EndstufenAP.png}
             \end{minipage}
\vspace{2mm}\hrule
        \subsection{FET-Transistor}
            \subsubsection{JFET Junction Field Effect Transistor}
            \begin{minipage}[T]{8cm}
                JFET als {\bf Konstantstromquelle}:\\
                ben\"otigter Strom \fbox{$I_D = \frac{U_{GS}}{R_S}$}\\
                $U_{GS}$ entsprechend ben\"otigem Strom aus Kennlinie lesen\\
                bei der Pinch-off-Grenze (Abschn\"urgrenze) sperrt der JFET
            \end{minipage}
            \begin{minipage}[T]{3.4cm}
                \includegraphics[height=4cm]{./bilder/JFETCCQuelle.png}
            \end{minipage}
            \begin{minipage}[T]{6cm}
                \includegraphics[height=4cm]{./bilder/JFETKennlinie.png}
            \end{minipage}
\hrule
            \subsubsection{MOSFET Metal Oxide Silicon Field Effect Transistor}
            \begin{minipage}[T]{6cm}
                \underline{Sperrbereich}\\\\
                $V_{GS} < V_{th} \to I_D = 0$\\
                typische $V_{th} = 0.5 … 1.5V$\\ 
                \includegraphics[height=4cm]{./bilder/MOSFETSperrbereich.png}
            \end{minipage}
            \begin{minipage}[T]{6cm}
                \underline{Widerstands-/Triodenbereich (linear)}\\\\
                $V_{GS} > V_{th} $ und $V_{DS}>0$\\
                $\to I_D$ fliesst\\
                \includegraphics[height=4cm]{./bilder/MOSFETLinBereich.png}
            \end{minipage}
            \begin{minipage}[T]{6cm}
                \underline{S\"attigungs-/Pentodenbereich}\\\\
                $V_{DS} > V_{GS} - V_{th}$ Kanal wird abgeschn\"urt\\
                \includegraphics[height=4cm]{./bilder/MOSFETSaettBereich.png}
            \end{minipage}\\
            
            \subsubsection{Steuerkennlinien von verschiedenen MOSFETs}
            \begin{minipage}[T]{9cm}
                \includegraphics[height=2.4cm]{./bilder/MOSFETSteuerkennlinien.png}
            \end{minipage}
            \begin{minipage}[T]{9cm}
                a) n-Kanal Anreicherungs-Typ (selbstsperrend)\\
                b) n-Kanal Verarmungs-Typ (selbstleitend)\\
                c) p-Kanal Anreicherungs-Typ (selbstsperrend)\\
                d) p-Kanal Verarmungs-Typ (selbstleitend)\\
            \end{minipage}
                
\newpage            
            \subsubsection{Berechnung des Drainstromes $I_D$}
            \begin{minipage}{13cm}
                $I_D = \begin{cases}
                0                       & $f\"ur $ V_{GS} \leqq V_{th}\\
                {\bf Sperrbereich}\\\\
                
                \beta\cdot(V_{GS} - V_{th})^2\cdot(1 + \lambda\cdot V_{DS})    &  $f\"ur $ 0\leqq V_{GS} - V_{th} \leqq V_{DS}\\
                {\bf Pentodenbereich} $ PB S\"attigungsbereich $\\\\
                
                \beta\cdot(2(V_{GS}-V_{th})V_{DS} - V_{DS}^2)\cdot(1 + \lambda\cdot V_{DS}) &  $f\"ur $ 0\leqq V_{GS} - V_{th} \geqq V_{DS}\\
                {\bf Triodenbereich} $ TB linearer Bereich$
                
                \end{cases}$\\
                
                \begin{tabular}[t]{l l l}
                    $b$: Kanalbreite & $L$: Kanall\"ange & $\mu_n$: Leitf\"ahigkeit Kanal\\
                    $\epsilon_{ox}$: Dielektrizit\"at Oxidschicht & $\lambda$: Pinch-off-Konstante & $d_{ox}$: Oxiddicke\\
                    $V_{th}$: Threshold-Spannung & $\beta$: Steilheitsparameter & $K$: Steilheitskoeffizient
                \end{tabular}
            \end{minipage}
            \vrule \hspace{0.1cm}
            \begin{minipage}[T]{6cm}
                \includegraphics[width=6cm]{./bilder/MOSFET_IU_Kennlinie.png}\\
                \vspace{0.8cm}\\
                Steilheitsparameter \hspace{1mm}\fbox{$\beta = \frac{K}{2} = \frac{\mu_n \epsilon_{ox}}{2d_{ox}}\frac{b}{L}$}
            \end{minipage}\\
            
            \subsubsection{Kleinsignalmodell des MOSFET}
            \begin{minipage}[T]{10.5cm}
                Steilheit
                \hspace{29.3mm}\fbox{$S = g_m = 2\beta(V_{GS}-V_{th}) = \sqrt{4\beta\cdot I_D}$}\\
                Ausgangsleitwert
                \hspace{15.9mm}\fbox{$g_d = \beta\lambda(V_{GS}-V_{th})^2$}\\
                Drain-Source-Widerstand
                \hspace{3mm}\fbox{$r_{DS} = \frac{1}{\lambda\cdot I_{D0}}$}\\
                Gate-Source-Spannung
                \hspace{7mm}\fbox{$V_{GS} \cong \sqrt{\frac{I_D}{\beta}} + V_{th}$}
            \end{minipage}
            \begin{minipage}[T]{3.5cm}
                \includegraphics[width=3cm]{./bilder/MOSFET_Aufbau.png}
            \end{minipage}
            \begin{minipage}[T]{5cm}
                \includegraphics[width=5cm]{./bilder/MOSFET_Ersatzsch.png}
            \end{minipage}
\vspace{1mm}\hrule
            \subsubsection{Differenzverst\"arker}
            \begin{minipage}[T]{14cm}
                Ausgangswiderstand
                \hspace{10.6mm}\fbox{$r_{out} = R//r_{DS}$}\\
                Verst\"arkung
                \hspace{23.3mm}\fbox{$A_D = -S \cdot (R_D//r_{DS}) \approx -S \cdot R_D$}\\
                Ausgansdifferenzspannung
                \hspace{1.7mm}\fbox{$V_{out_{diff}} = A \cdot V_{in_{diff}}$}
                
            \end{minipage}
            \begin{minipage}[T]{5cm}
                \includegraphics[height=4cm]{./bilder/MOSFET_Diffamp.png}
            \end{minipage}
\hrule
\hrule

	\section{Referenzspannungsquellen}
        \subsection{Dioden-Referenz}
            \begin{minipage}[T]{8.5cm}
                Referenzspannung
                \hspace{14mm}\fbox{$V_{Ref} = \frac{k_B\cdot T}{q}\cdot\ln\left(\frac{I}{I_S}\right)$}\\
                Sensitivit\"at
                \hspace{24.2mm}\fbox{$S = \frac{1}{\ln\left(\frac{I}{I_S}\right)}\ll 1$}
            \end{minipage}
            \begin{minipage}{4.5cm}
                $I_S$=S\"attigungssperrstrom\\
                $k_B$= Boltzmann-Konstante\\
                $V_T$=Thermospannung
            \end{minipage}
            \begin{minipage}{6cm}
                \includegraphics[width=6cm]{./bilder/DiodenRef.png}\\
            \end{minipage}\\
            
        \subsection{Bandgap-Referenz}
            \begin{minipage}[T]{10cm}
                Ziel der Bandgap-Referenz: Die enthaltene PTAT-Stromquelle hat einen positiven und die enthaltene als Transistor realisierte Diode einen negativen Temperaturkoeffizienten. Dadurch kann eine nahezu temperaturunabh\"angige Spannungsreferenz mit $V_{ref}\approx 1.2V$ realisiert werden.
            \end{minipage}
            \begin{minipage}[T]{5cm}
                \includegraphics[width=5cm]{./bilder/BandgapBlocksch.png}
            \end{minipage}
            \begin{minipage}[T]{4cm}
                \includegraphics[width=4cm]{./bilder/Bandgap.png}
            \end{minipage}
\newpage
	\section{Spannungsregler}
        {\bf{Qualit\"atsmasse}}\\
            \begin{minipage}[t]{6cm}
                Ausg.-spannung bestimmt durch:\\
                relativen Stabilisierungsfaktor $S'$\\
                \vspace{0.5mm}{$S' = \frac{\frac{\Delta U_e}{U_e}}{\frac{\Delta U_a}{U_a}}$}
            \end{minipage}
            \begin{minipage}[t]{6cm}
                Temperaturverhalten:\\
                Temperaturkoeffizient $TK_U$\\
                \vspace{0.5mm}{$TK_U = \frac{1}{U_a}\frac{dU_a}{dT}$}
            \end{minipage}
            \begin{minipage}[t]{6cm}
                Lastabh\"angigkeit:\\
                dynamischer Ausgangswiderstand $r_a$\\
                \vspace{0.5mm}{$r_a = \frac{dU_a}{dI_a}$}
            \end{minipage}\\    
        \subsection{Linearer Regler}
            \begin{minipage}[T]{14cm}
                Ausgangsspannung
                \hspace{13mm}\fbox{$V_{out} = \left(1+\frac{R_1}{R_2}\right) \cdot V_{ref}$}\\
            \end{minipage}
            \begin{minipage}{5cm}
                \includegraphics[height=1.5cm]{./bilder/ReglerLinear.png}
            \end{minipage}\\
            
        \subsection{Einstellbarer Regler}
        \begin{minipage}[T]{14cm}
            Ausgangsspannung
            \hspace{13mm}\fbox{$V_{out} = V_{ref} + R_2 \cdot\left(\frac{V_{ref}}{R_1} + I_{adj}\right)  \approx\left(1+\frac{R_2}{R_1}\right) \cdot V_{ref}$}\\
        \end{minipage}
        \begin{minipage}{5cm}
            \includegraphics[height=2.5cm]{./bilder/ReglerEinstellbar.png}
        \end{minipage}\\
        
        \subsection{Aufw\"artswandler (Step up, Boost)}
        \begin{minipage}[T]{14cm}
            Lade-Phase (S geschlossen)
            \hspace{0.5mm}\fbox{$\Delta I_{L\_on} = \frac{1}{L}V_{in}\cdot t_{on}$}\\  
            Entlade-Phase (S offen)
            \hspace{5.8mm}\fbox{$\Delta I_{L\_off} = \frac{1}{L}(V_{in}-V_{out})\cdot t_{off}$}\\  
            Gleichgewicht
            \hspace{21mm}\fbox{$\Delta I_{L\_on} =-\Delta I_{L\_off}$}\\  
            Ausgangsspannung
            \hspace{13mm}\fbox{$V_{out} = V_{in}\cdot \left(1+\frac{t_{on}}{t_{off}}\right)$}\\
        \end{minipage}
        \begin{minipage}{5cm}
            \includegraphics[width=4cm]{./bilder/ReglerStepUpStromverlauf.png}
        \end{minipage}\\
        
        \subsection{Abw\"artswandler (Step Down, Buck)}
        \begin{minipage}[T]{14cm}
            Lade-Phase (S geschlossen)
            \hspace{0.5mm}\fbox{$\Delta I_{L\_on} = \frac{1}{L}\int_{0}^{T_i}{(V_{in}-V_{out})dt} = \frac{1}{L}(V_{in}-V_{out})\cdot T_i$}\\  
            Entlade-Phase (S offen)
            \hspace{5.8mm}\fbox{$\Delta I_{L\_off} = \frac{1}{L}\int_{T_i}^{T_S}{(V_{out}+V_{D_f})dt}= \frac{1}{L}(V_{out}+V_{D_f})(T_S-T_i)$}\\
            Ausgangsspannung
            \hspace{13mm}\fbox{$V_{out} = \frac{T_i}{T_S}\cdot V_{in} - \left(1-\frac{T_i}{T_S}\right) \cdot V_{D_f} \approx \frac{T_i}{T_S}\cdot V_{in}$}\\
        \end{minipage}
        \begin{minipage}{5cm}
            \includegraphics[width=5cm]{./bilder/ReglerStepDown.png}
        \end{minipage}\\
        
        \subsection{Invertierender Wandler (Buck-Boost)}
        \begin{minipage}[T]{8cm}
            Ausgangsspannung
            \hspace{13mm}\fbox{$V_{out} = -\frac{T_i}{T_S}\cdot V_{in}$}\\
        \end{minipage}
        \begin{minipage}{6cm}
            $T_i$ = Einschaltzeit\\
            $T_s$ = Ausschaltzeit
        \end{minipage}
        \begin{minipage}{5cm}
            \includegraphics[width=4cm]{./bilder/ReglerInvert.png}
        \end{minipage}\\
        
        \subsection{Ladepumpen (Charge-Pumps)}
        \begin{minipage}[T]{14cm}
            \hspace*{43.3mm}\fbox{$\Delta V_{out} = \frac{\Delta Q_1}{C_1+C_L} = \frac{C_1(V_{in}-V_{out})}{C_1+C_L} = \frac{V_{in}-V_{out}}{1+\frac{C_L}{C_1}}$}\\
            \hspace*{43.3mm}\fbox{$\Delta Q = C_L \cdot \Delta V_{out} = \frac{C_1 C_L}{C_1+C_L}(V_{in}-V_{out})$}\\
            \hspace*{43.3mm}\fbox{$I_{out} = \frac{\Delta Q}{T_S} = \frac{C_1 C_L}{C_1+C_L}\frac{V_{in}-V_{out}}{T_S}$}\\
            \hspace*{43.3mm}Ver\"anderung der Ausgangsspannung durch ver\"andern der\\
            \hspace*{43.3mm}Schaltfrequenz
        \end{minipage}
        \begin{minipage}{5cm}
            \includegraphics[width=5cm]{./bilder/ReglerChargePump.png}
        \end{minipage}\\   
\newpage

    \subsection{Wichtiges zu Spule / Kondensator}
        Energie im Mag.-Feld
        \hspace{9mm}\fbox{$W_m = \frac{1}{2} \cdot L \cdot I^2 = \frac{1}{2}\cdot N \cdot I \cdot \Phi$}\\
        Energie im El.-Feld
        \hspace{12.5mm}\fbox{$W_E = \underbrace{\frac{1}{2} \cdot C \cdot U^2}_{U = konst}  = \frac{1}{2} \cdot Q \cdot U = \underbrace{\frac{1}{2}\cdot\frac{Q^2}{C}}_{Q = konst}$}\\
        Spule
        \hspace{34mm}\fbox{$u_L(t) = L \cdot \frac{di(t)}{dt}$}\\
        \hspace*{43.5mm}\fbox{$i_L(t) = \frac{1}{L}\int\limits_0^t u(\tau)d\tau + i(0)$}\\
        Kondensator
        \hspace{22.7mm}\fbox{$u_C(t) = \frac{1}{C}\int\limits_0^t i(\tau)d\tau + u(0)$}\\
        \hspace*{43.5mm}\fbox{$i_C(t) = C \cdot \frac{du(t)}{dt}$}\\
        
    \section{Oszillatoren}
        \subsection{Oszillator\"ubersicht}
            \includegraphics[height=3cm]{./bilder/OszillatorKlassifikation.png}
            
        \subsection{Ringoszillator}
            \begin{minipage}[T]{8cm}
                Oszillatorfrequenz
                \hspace{14mm}\fbox{$f_{OSC} = \frac{1}{2\cdot n \cdot t_g}$}
            \end{minipage}
            \begin{minipage}[T]{6cm}
                $t_g$: Time-Delay pro Inverter\\
                $n$: ungerade Anzahl Inverter\\\\
                Schaltung ist jedoch nicht vernachl\"assigbar abh\"angig von Technologie, Speisung und Temperatur
            \end{minipage}
            \begin{minipage}{4cm}
                \hspace*{5mm}
                \includegraphics[width=4cm]{./bilder/Ringoszillator.png}
            \end{minipage}\\
            
        \subsection{LC-Oszillator}
        \begin{minipage}[T]{15cm}
            Oszillatorfrequenz
            \hspace{14mm}\fbox{$\omega_0 = \frac{1}{\sqrt{LC}}$}\\
        \end{minipage}
        \begin{minipage}{4cm}
            \includegraphics[height=2cm]{./bilder/LC_Oszillator.png}
        \end{minipage}\\

        \subsection{CMOS-Inverter mit Quarz}
        \includegraphics[height=3cm]{./bilder/QuarzOszillator.png}
        
\end{document}

